\documentclass[10pt,a4paper,sans]{moderncv}

%% ModernCV themes
\moderncvstyle{classic}
\moderncvcolor{black}
\renewcommand{\familydefault}{\sfdefault}
\nopagenumbers{}

%% Character encoding
\usepackage[utf8]{inputenc}

%% Adjust the page margins
\usepackage[scale=0.85]{geometry}
\usepackage{tabularx}
%% Personal data
\name{Coen}{D. Needell}
\title{Publications, Projects, and Presentations}
\mobile{+1~(970)~456~2411}
\email{coen@needell.org}
\social[github]{SoyBison}
\social[linkedin]{coen-needell-b4503216a}
\homepage{coen.needell.org}
\photo[64pt][1pt]{avatar}


%%------------------------------------------------------------------------------
%% Content
%%------------------------------------------------------------------------------

\begin{document}
\makecvtitle
\section{Publications}

\cventry{2023}{Mainstream Media Producing and Distributing Misinformation Through Stenography and ``Opinions''}{}{\linebreak David Rothschild, Elliot E. Pickens, Tin Orešković, Katelyn Morrison, Markus Mobius, Coen D. Needell, Jared Katzman, Duncan J. Watts}{}{Forthcoming}

\cventry{2023}{Modeling Framing in News Coverage about the U.S. Economy}{}{\linebreak Maria Leonor Pacheco, Coen D. Needell, Elliot E. Pickens, David Rothschild}{}{Forthcoming}

\cventry{2023}{The News Observatory: A collection and Storage System for Past and Future News Media Content}{}{\linebreak Coen D. Needell, Elliot E. Pickens, David Rothschild}{}{Forthcoming}

\cventry{2023}{Search Conversion Journeys and Efficient Advertising Opportunities}{}{\linebreak Coen D. Needell, David Rothschild}{\textit{Marketing Science}}{Submitted}

\cventry{2021}{\href{https://doi.org/10.1007/s42113-022-00126-5}{Embracing New Techniques in Deep Learning for Estimating
Image Memorability}}{}{\linebreak Coen D. Needell, Wilma A. Bainbridge}{\textit{Computational Brain \& Behavior}}{Published}

\section{Presentations}

\cventry{2023}{The News Observatory: A collection and Storage System for Past and Future News Media Content}{}{\linebreak Coen D. Needell, Elliot E. Pickens, David Rothschild, Duncan J. Watts}{\textit{International Conference on Computational Social Science}}{Poster}

\cventry{2021}{Memorability: A Stimulus-Centric Framework for Analyzing Memory Performance}{}{\linebreak Wilma A. Bainbridge, Paige Hanson, Max Kramer, Coen D. Needell, Xinyue Li}{\textit{Interdisciplinary Graduate Conference, UChicago}}{Panel}

\cventry{2021}{Embracing New Techniques in Deep Learning for Predicting Image Memorability}{}{\linebreak Coen D. Needell, Wilma A. Bainbridge}{\textit{Annual Meeting of the Vision Sciences Society}}{Poster}

\section{Selected Projects}

\cventry{2023 -- }{ResMemVox}{}{}{}{In collaboration with the Brain Bridge Lab at the University of Chicago. ResMemVox is a deep learning based model for estimating the memorability of short voice snippets.}

\cventry{2022 -- }{\href{https://coen.needell.org/project/news_observatory/}{News Observatory}}{}{}{}{Using cloud computing, optical character recognition, networking techniques, and other mixed methods to create an ongoing dataset of news website publications, as well as automated systems to manage and validate data.}

\cventry{2022 -- }{Hugo Data Visualizations}{}{}{}{Using Hugo, D3.JS, and other assorted JS libraries to create a framework for fast publication of interactive and live-updating social scientific analyses.}

\cventry{2021 -- 2022}{Conversion Journeys}{}{}{}{Using natural language processing techniques on behavioral data to examine how consumers are converted to purchasing particular durable goods. In collaboration with Bing.}

\cventry{2021 -- 2022}{Project Ratio: Framing}{}{}{}{A project in collaboration with Project Ratio: examining how we perceive modern news media.}

\cventry{2021 --}{Speed of News}{}{}{}{Using data from the wayback machine and archive.org, this is a project focused on examining how major publishers publication behavior has changed over time. This is especially focused on the concept of the 24-hour news cycle, and how internet news delivery differs from print media. With Microsoft Research.}

\cventry{2021 --}{R56}{}{}{}{In collaboration with the Memory Lab at the University of Chicago. The R56 is a battery of tests for mild cognitive impairment that can be administered over the internet, through the use of modern web frameworks and systems design.}

\cventry{2021}{\href{https://www.coeneedell.com/post/gammatones}{High Speed Gammatone Cepstral Decomposition}}{}{}{}{Connected to Ongaku, an implementation of the gammatone cepstral decomposition for OpenCL. This would allow it's use in real time applications or large scale machine learning pipelines.}

\cventry{2020 -- 2021}{\href{https://www.coeneedell.com/project/memnet/}{Deep Learning and Computer Vision for Memorability}}{}{}{}{A project to create a better computer vision model for predicting the memorability of an image. This started with investigating the current standard MemNet, and has moved beyond into developing new models including the now completed \textit{ResMem}.}

\cventry{2020}{\href{https://www.coeneedell.com/project/computational_patahistory/}{Computational Rupahistory}}{}{}{}{An ongoing side project to see how agent based simulations of territory-controlling groups interact in a simulated world. Currently on the back-burner, though some progress has been made in creating a cellular automata to generate a world map on a hex-grid.}

\cventry{2020}{\href{https://www.coeneedell.com/project/bandcamp_analysis/}{Bandcamp Album Covers}}{}{}{}{A project to investigate how indie musicians use visual signs to indicate their subgenre. Leverages Natural Language Processing techniques like Latent Dirichlet Allocation to analyze color usage in album cover images.}

\cventry{2019 -- 2020}{\href{https://www.coeneedell.com/project/ongaku/}{Ongaku}}{}{}{}{A system for creating musical playlists based on feature analysis. Leverages gammatone cepstral coefficients (a system for mimicking neural signals from the ear to the brain) and manifold learning techniques to create a psuedo-euclidean space for musical tracks. Shapes in the song-space can then be drawn to define playlists.}

\cventry{2019}{\href{https://www.coeneedell.com/project/fluxx-for-robots/}{Fluxx for Robots}}{}{}{}{An Artifical Intelligence Learning environment for the tabletop card-game Fluxx by Looney Labs. Has both a human-motivated interface and a machine-motivated interface. Intended for research on machine learning methods for complex and incomplete-information games.}

\end{document}
